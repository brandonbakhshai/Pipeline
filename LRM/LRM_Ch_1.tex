\documentclass[./LRM_main.tex]{subfiles}
\begin{comment}
If you want a box around your answer and that answer is an
equation then use \boxed{$$ equation $$} 

if you want to indent a block of text:
\begin{adjustwidth}{cm of right indent}{cm of left indent}
% paragraph to be indented
\end{adjustwidth}

if you just want one indent for one line 
use \indent per intended indent per line

A sections numbers automatically, so if the number of 
the problem is out of order it would be easier to 
just indent and bold the sections and subsections
and not use the \section{} kind of commands

\newpage makes a new page

$normal math mode$
$$Special math mode$$

to include an image use
\includegraphics{image_name}
image_name is the file name (.png) without the extension. The file
name cannot have any spaces or any periods other than the one before
the file extension.

To include a codeblock use
\begin{lstlisting}
ExampleCode(blah, blah)
{
	it does tabbing and everything;
	for (coloring of major languages like java){
		add the following to the \lstset tuple:
			language=<name_of_language>;
	}
}
\end{lstlisting}

\end{comment}


\begin{document}

%\tableofcontents

%\thispagestyle{empty}
%\newpage
% If you want to change how the subsubsection's are numbered
%\renewcommand{\thesubsection}{\thesection.\alph{subsection}.} 

%\setcounter{page}{0}
\chapter{Introduction}
Concurrent programming has become a very important paradigm in 
modern times, with mainstream languages such as Java, Python, and C++ offering 
concurrent programming mechanisms as part of their APIs. However, they tend to 
be complicated - sometimes necessarily - and invite a host of additional 
concerns like atomicity. Node.js has emerged as a framework with a unique approach toward asynchronous 
programming - the single-threaded (but not really) asynchronous programming. 
The event-driven architecture of Node.js and nonblocking I/O API makes it a 
perfect fit for backend web development.

Our intent with Pipeline is to build a simple language that encompasses these features from 
Javascript and the Node.js framework - easy asynchronous programming using the 
event-driven architecture and a speedy I/O API.

\section{About Pipeline}
Pipeline is a structured inperative C-style language that incorporates the asynchronous programming model of Node.js in the form of a pipeline. Pipeline expands on the idea of Javascript's promises, and maked this concept central to the design of the language in the form of a pipeline. A pipeline allows the programmer to chain functions together that must run synchronously and handle them asynchronously from the body of code in which it resides -- a manner similar to Promises from Javascript, except with a more convienient syntax. 

\section{tokens}
There are five classes of tokens:
identifiers, keywords, string literals,operators, other separators.\\
The following "white space" characters are ignored except they separate tokens: blanks, tabs, newlines, and comments.\\
\end{document}

