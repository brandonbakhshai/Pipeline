\documentclass[./LRM_main.tex]{subfiles}
\begin{comment}
If you want a box around your answer and that answer is an
equation then use \boxed{$$ equation $$} 

if you want to indent a block of text:
\begin{adjustwidth}{cm of right indent}{cm of left indent}
% paragraph to be indented
\end{adjustwidth}

if you just want one indent for one line 
use \indent per intended indent per line

A sections numbers automatically, so if the number of 
the problem is out of order it would be easier to 
just indent and bold the sections and subsections
and not use the \section{} kind of commands

\newpage makes a new page

$normal math mode$
$$Special math mode$$

to include an image use
\includegraphics{image_name}
image_name is the file name (.png) without the extension. The file
name cannot have any spaces or any periods other than the one before
the file extension.

To include a codeblock use
\begin{lstlisting}
ExampleCode(blah, blah)
{
	it does tabbing and everything;
	for (coloring of major languages like java){
		add the folloing to the \lstset tuple:
			language=<name_of_language>;
	}
}
\end{lstlisting}

\end{comment}


\begin{document}

%\tableofcontents

%\thispagestyle{empty}
%\newpage
% If you want to change how the subsubsection's are numbered
%\renewcommand{\thesubsection}{\thesection.\alph{subsection}.} 

%\setcounter{page}{0}
\chapter{Data Types}



\section{Primitive Typpes}
These are the following primitive data types in Pipeline:
\begin{itemize}
    \item int -  Standard 32-bit (4 bytes) signed integer. It can take any value in the range [-2147483648, 2147483647].
    \item float - Single precision floating point number, that occupies 32 bits (4 bytes) and its significand has a precision of 24 bits (about 7 decimal digits). 
    \item char - An 8-bit (1 byte) ASCII character. Extended ASCII set is included, so we use all 256 possible values.
    \item pointer - A 64-bit pointer that holds the value to a location in memory; very similar to those found in C.
\end{itemize}

\section{Compound Types}
\begin{itemize}
    \item String - objects that represent sequences of characters. 
    \item List - A data structure that lets you store one or more elements (of a particular data type) consecutively in memory. The elements are indexed beginning at position zero. 
    \item Struct - A structure is a programmer-defined data type made up of variables of other data types (possibly including other structure types).
\end{itemize}

Examples with compound data types:
\begin{lstlisting}
                      a = [1,2,3,4,5];
		              c = [ [1,2,3,4],[5,6,7,8]];
		              d = []
		              
		              struct foo {
			               int bar;
			               string bar1;
			               bar2 = []
                     }

\end{lstlisting}

%\subsection{subsection}
%\subsubsection{subsubsection}
\section{Complex Types}

\end{document}

