\documentclass[./LRM_main.tex]{subfiles}
\begin{comment}
If you want a box around your answer and that answer is an
equation then use \boxed{$$ equation $$} 

if you want to indent a block of text:
\begin{adjustwidth}{cm of right indent}{cm of left indent}
% paragraph to be indented
\end{adjustwidth}

if you just want one indent for one line 
use \indent per intended indent per line

A sections numbers automatically, so if the number of 
the problem is out of order it would be easier to 
just indent and bold the sections and subsections
and not use the \section{} kind of commands

\newpage makes a new page

$normal math mode$
$$Special math mode$$

to include an image use
\includegraphics{image_name}
image_name is the file name (.png) without the extension. The file
name cannot have any spaces or any periods other than the one before
the file extension.

To include a codeblock use
\begin{lstlisting}
ExampleCode(blah, blah)
{
	it does tabbing and everything;
	for (coloring of major languages like java){
		add the folloing to the \lstset tuple:
			language=<name_of_language>;
	}
}
\end{lstlisting}

\end{comment}


\begin{document}

%\tableofcontents

%\thispagestyle{empty}
%\newpage
% If you want to change how the subsubsection's are numbered
%\renewcommand{\thesubsection}{\thesection.\alph{subsection}.} 

%\setcounter{page}{0}
\chapter{Lexical Conventions}

Pipelines is a free form language, i.e the position of characters in the program is insignificant. The parser will discard whitespace characters such as '', '\textbackslash t', and '\textbackslash n'. \\

\section{Identifiers}
Identifiers for Pipeline will be defined in the same way as they are in most other languages; any sequence of letters and numbers without whitespaces and is not a keyword will be parsed as an identifier. Identifiers cannot begin with a number.The variables are declared : vartype varname; //

The regular expression defining identifiers is as follows:  

\begin{lstlisting}
["a"-"z" "A"-"Z"]["a"−"z" "A"−"Z" "0"−"9" "_ "]∗
\end{lstlisting}

These are examples definitions:

\begin{lstlisting}
int 2number int; /∗ not a valid identifier declaration ∗/ 
float number; /∗ valid ∗/ 
int number1; /∗ valid ∗/
\end{lstlisting}

\section{Literals}
Literals are sequence of numbers, which may be identified with the regular expression : 
\begin{lstlisting}
["0"−"9"]\mbox{*}"."["0"−"9"]+ (∗ Float ∗)
["0"−"9"]+ (∗ Int ∗) 
\end{lstlisting}

\section{Tokens}
These are the list of tokens used in pipeline:\\
\begin{lstlisting}

| "(" { LPAREN } 
| ")" { RPAREN } 
| "{" { LBRACE } 
| "}" { RBRACE } 
| ";" { SEMI } 
| "." { NAMESPACE }
| ":" { COLON } 
| "," { COMMA } 
| "+" { PLUS } 
| "−" { MINUS }
| "∗" { TIMES } 
| "%" { MOD } 
| ">>" { RSHIFT } 
| "<<" { LSHIFT } 
| "/" { DIVIDE } 
| "=" { ASSIGN } 
| "==" { EQ }
| "!=" { NEQ } 
| "<" { LT } 
| "<=" { LEQ } 
| ">" { GT } 
| ">=" { GEQ } 
| "!" { NOT } 
| "if" { IF } 
| "else" { ELSE } 
| "elif" { ELIF } 
| "for" { FOR } 
| "return" { RETURN } 
| "int" { INT }
| "float" { FLOAT } 
| "char" { CHAR }
| "pipe" {PIPE}
| "@" { POINTER } 
| "&" { AMPERSAND } 
| "fun" { FUNCTION } 
| "pipe" {PIPELINE}
| "void" { VOID } 
| "struct" { STRUCT } 
| "string" { STRING } 
| "break" { BREAK } 
| "coupling" {COUPLING}
 

\end{lstlisting}


\section{Punctuation}
\subsubsection{Semicolon}
As in C, the semicolon ‘;’ is required to terminate any statement in Pipeline
\begin{lstlisting}
statement SEMI
\end{lstlisting}

\subsubsection{Braces}
In order to keep the language free-format, braces are used to separate blocks. These braces are required even for single-statement conditional and iteration loops. 
\begin{lstlisting}
LBRACE statements RBRACE
\end{lstlisting}


\subsubsection{Comments }
Comments are initiated with ‘/* ‘ and closed with ‘*/’ and cannot be nested.}
%\subsection{subsection}
%\subsubsection{subsubsection}
\end{document}

