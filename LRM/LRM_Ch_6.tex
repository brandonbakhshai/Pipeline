\documentclass[./LRM_main.tex]{subfiles}
\begin{comment}
If you want a box around your answer and that answer is an
equation then use \boxed{$$ equation $$} 

if you want to indent a block of text:
\begin{adjustwidth}{cm of right indent}{cm of left indent}
% paragraph to be indented
\end{adjustwidth}
\begin{adjustwidth}{1cm}{}

\end{adjustwidth}


if you just want one indent for one line 
use \indent per intended indent per line

A sections numbers automatically, so if the number of 
the problem is out of order it would be easier to 
just indent and bold the sections and subsections
and not use the \section{} kind of commands

\newpage makes a new page

$normal math mode$
$$Special math mode$$

to include an image use
\includegraphics{image_name}
image_name is the file name (.png) without the extension. The file
name cannot have any spaces or any periods other than the one before
the file extension.

To include a codeblock use
\begin{lstlisting}
ExampleCode(blah, blah)
{
	it does tabbing and everything;
	for (coloring of major languages like java){
		add the following to the \lstset tuple:
			language=<name_of_language>;
	}
}
\end{lstlisting}

\end{comment}


\begin{document}

%\tableofcontents

%\thispagestyle{empty}
%\newpage
% If you want to change how the subsubsection's are numbered
%\renewcommand{\thesubsection}{\thesection.\alph{subsection}.} 

%\setcounter{page}{0}
\chapter{Functions}
\section{Anatomy of a Function}
A function in pipeline is defined with the keyword \texttt{fun} followed by the function name, the parameters and then the return type or types if applicable. The function definition must contain these elements in order to be a viable function, and the body of the function must be enclosed in curly braces ('\{' '\}'). In general the function definition has the following Grammar and syntax:\\
\begin{lstlisting}
fun function_name(P_type_1 p_name_1, ..., P_type_n p_name_n)(ret_type)
{
	<body of code>
}
\end{lstlisting}
A function need not take any parameters, nor need it return any values.
\subsection{Declaration}
In Pipeline, as in C, the function must exist before it is called to be used. Meaning, you cannot call any function within any other function unless it has been explicitly stated previous to that function was defined. A function declaration allows the programmer to declare a function exists before he/she has defined the function itself. A function declaration is almost identical to the above function definition, except that instead of curly braces and a code body, there is only a semicolon:
\begin{lstlisting}
fun function_name(P_type_1 p_name_1, ..., P_type_n p_name_n)(ret_type)
{
	<body of code>
}
\end{lstlisting}
\section{Grammar}
\begin{lstlisting}
fun-definition:
	fun declarator param-list_opt type-qualifier_opt compund-statement

fun-declaration:
	fun declarator param-list_opt type-qualifier_opt;

param-list_opt:
	( param_list )
	( )

type-qualifier_opt:
	( type-specifier )
	( )

param-list:
	declaration-specifiers declarator
	param-list , declaration-specifiers declarator


\end{lstlisting}
\subsection{Parameters and Return values}
Pipeline is a pass by value language, meaning that whenever a value is passed to another function or from another function, it is merely a copy. This is why Pipeline provides pointers like C, so that they can pass and manipulate a given object in memory, and not a copy of that object that keeps the original intact. Therefore the scope of any variable is the function in which is was declared, and in order for it to exist external to that function a pointer to its memory location must be provided.
\subsection{The "main" function}
The main function is the function that is executed at run time. The main function has as parameters char *string and int argc, which are related to the command-line arguments provided at run-time. The argv variable holds a pointer to the array of the strings typed by the user at run-time, the first of which is always the name of the executable, and the argc variable is the number of those arguments including the executable name.
\section{Scope}
The scope of a variable is from the point it was declared, until the end of the translation unit in which it resides. By translation unit, I am referring to control flow code bodies, functions, pipelines and ,in the case of a global variable, the program itself. The scope a parameter is from the point at which the function code block starts, until the end of the function.




%\subsection{subsection}
%\subsubsection{subsubsection}

\end{document}

