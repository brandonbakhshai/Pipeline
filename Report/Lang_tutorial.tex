\documentclass[./Report_main.tex]{subfiles}
\begin{comment}
If you want a box around your answer and that answer is an
equation then use \boxed{$$ equation $$} 

if you want to indent a block of text:
\begin{adjustwidth}{cm of right indent}{cm of left indent}
% paragraph to be indented
\end{adjustwidth}

if you just want one indent for one line 
use \indent per intended indent per line

A sections numbers automatically, so if the number of 
the problem is out of order it would be easier to 
just indent and bold the sections and subsections
and not use the \section{} kind of commands

\newpage makes a new page

$normal math mode$
$$Special math mode$$

to include an image use
\includegraphics{image_name}
image_name is the file name (.png) without the extension. The file
name cannot have any spaces or any periods other than the one before
the file extension.

To include a codeblock use
\begin{lstlisting}
ExampleCode(blah, blah)
{
	it does tabbing and everything;
	for (coloring of major languages like java){
		add the folloing to the \lstset tuple:
			language=<name_of_language>;
	}
}
\end{lstlisting}

\end{comment}


\begin{document}

%\tableofcontents

%\thispagestyle{empty}
%\newpage
% If you want to change how the subsubsection's are numbered
%\renewcommand{\thesubsection}{\thesection.\alph{subsection}.} 

%\setcounter{page}{0}
\chapter{Language Tutorial}
\section{Setup}
\section{Program composition}
All statements in a pipeline program must be in one of the following blocks(function, pipe, anonymous block) with global variables as exception.\\
\begin{lstlisting}
    global int x = 10; // global variable not in a block, allowed
    
    
    
    function string hello(){
            return "hello" // statement in a function block, allowed
    }
    
    
    {
    int a;    //statements in anonymous block, allowed
    print_a
    }
    
    
    pipe{
        print_str("stmt in a pipe")// statement in a pipe block, allowed
    }
    
    
    print_int(x)// illegal, statement that's not in any block and not a global variable declaration.
\end{lstlisting}
\section{Variable declaration and assignment}
\subsection{primitive types}
pipeline has 4 primitive types:
\begin{itemize}
    \item int
    \item float
    \item string
    \item boolean
\end{itemize}
variable declaration for primitive types works as follow:
\begin{lstlisting}
    
int a; // type var-name for variable declaration
    
int b = 1;// type var-name expr, for variable declaration and initialization
    
a = 10; //variable assignment
\end{lstlisting}
\subsection{string manipulation}
String type supports the following operations:\\
\begin{itemize}
    \item len
    \item sub
    \item cmp
    \item concat(with \$)
\end{itemize}
\begin{lstlisting}
    string a = "hello";//declare string variable a
    
    string b = " world";//declare string variable b
    
    int length;
    
    length = len(a); // length = 5
    
    bool str;
    
    str = cmp(a,b);//str = false
    
    str = sub(a,b);//str = false
    
    string c = a$b;// c = "hello world" $
\end{lstlisting}
\subsection{global variables}
global variables in pipeline is declared with the \textbf{global} keyword and assessable throughout the program.\\
Also, global variable declaration need not to be inside a block.\\
\begin{lstlisting}
    global int a = 10;//global variable a.
    function int foo(){
            return a;//global variable a is accessible.
    }
\end{lstlisting}
\subsection{List}
List is a compound type that can be applied all primitive types. List cannot be declared globally like primitive types\\
The pipeline list is a singly likened list supports the the following operation:\\
\begin{itemize}
    \item addleft
    \item addright
    \item popleft
    \item list\_free
    \item access
\end{itemize}
\textbf{list-declaration}
\begin{lstlisting}[]
    int a[]; //int-list declaration
    
    float b[]; //float-list declaration
    
    string c[]; //string-list declaration
    
    bool d[]; //boo-list declaration
\end{lstlisting}
\textbf{list-append}
\begin{lstlisting}
    addleft(a,1) //append 1 to the left of a, a = [1]
    addright(a,2) // append 2 to the right of a, a = [1,2]
    popleft(a) // pop the left-most element in a, a = [2]
\end{lstlisting}
\textbf{list-access}
\begin{lstlisting}
    // let a = [1,2,3,4,5]
    a[0] //1
    a[1] //2
    a[4] //5
\end{lstlisting}
If a list is declared, it must be freed in order to prevent memory leak in the program
\textbf{list-free}
\begin{lstlisting}
    list_free(a)//release the memory of a and prevent memory leak
\end{lstlisting}
\subsection{struct}
\subsection{file}
\section{control flow}
control flow in pipeline works similar as C, it has for and wile loop and if, else selection\\
\begin{lstlisting}
{
    int a; 
    for(a = 0;a<10;a=a+1){// a for loop that loops over 10 times
        print_int(a);
    }
    a = 0;
    while(a<10){// a while loop that loops over 10 times
        print_int(a);
    }
    if(true){// an if selection that will always print false
        print_bool(false);
    }else{
        print_bool(true);
    }
}
\end{lstlisting}
The following program prints the lyrics for 99-bottles of beers.\\
\begin{lstlisting}
 {
 int i;
 for (i = 99; i >0; i = i-1)
     {
 
     string wall = " bottles of beer on the wall";
     string beer = " bottles of beer.";
     string pass = "Take one down and paas it around, ";
     string nomore = "No more bottle of beer on the wall";
     if(i>1){
         print_int(i);
         print_str(wall$", ");
         print_int(i);
         print_str(beer$"\n");
         print_str(pass);
         print_int(i-1);
         print_str(wall$".\n\n");
         }
     else{
         print_int(i);
         print_str(wall$",");
         print_int(i);
         print_str(beer$"\n");
         print_str(pass);
         print_str(nomore$"\n\n");
         print_str(nomore $ ", no more bottle of beer"$"\n");
         print_str("Go to the store and buy some more, 99 bottles of beer on the wall.\n");
         }
     }
     }
\end{lstlisting}
\section{function}
while pipeline has many buildin functions, it also allows custom function declaration. The fallowing function calculates the gcd of two given numbers and returns the result.\\
\begin{lstlisting}
function int gcd(int a, int b){
    while(a != b){
        if(a>b){
            a = a-b;
        }else{
            b = b-a;
        }
    }
    return a;
}
{
int result = gcd(14,21)// result = 7
}
\end{lstlisting}
pipeline also supports recursive functions.\\
\begin{lstlisting}
function int fib(int x){//function the calculates the nth Fibonacci numbers
 
     if(x==0){
         return 0;
     }
     if (x==1){
         return 1;
     }
     return fib(x-1)+fib(x-2);
 }
 {
 int c = fib(4);
 
 print_int(c);
 }
\end{lstlisting}
\section{pipe}
\section{build-in functions}
build-in functions provided by the pipeline language:\\
\begin{itemize}
    \item print\_int(int) - print a int to stdout
    \item print\_float(float) - print a float to stdout
    \item print\_bool(bool) - print a bool to stdout
    \item print\_string(string) - print a string to stdout
    \item print\_error(string) - print a string to stderr
    \item sleep(int) - block the program for a given number of seconds
    \item exit(int) - exit the program with given status
\end{itemize}
%\subsection{Identifiers}
%\subsubsection{}
%\subsection{subsection}
%\subsubsection{subsubsection}
\end{document}

