\documentclass[./Report_main.tex]{subfiles}
\begin{comment}
If you want a box around your answer and that answer is an
equation then use \boxed{$$ equation $$} 

if you want to indent a block of text:
\begin{adjustwidth}{cm of right indent}{cm of left indent}
% paragraph to be indented
\end{adjustwidth}

if you just want one indent for one line 
use \indent per intended indent per line

A sections numbers automatically, so if the number of 
the problem is out of order it would be easier to 
just indent and bold the sections and subsections
and not use the \section{} kind of commands

\newpage makes a new page

$normal math mode$
$$Special math mode$$

to include an image use
\includegraphics{image_name}
image_name is the file name (.png) without the extension. The file
name cannot have any spaces or any periods other than the one before
the file extension.

To include a codeblock use
\begin{lstlisting}
ExampleCode(blah, blah)
{
	it does tabbing and everything;
	for (coloring of major languages like java){
		add the folloing to the \lstset tuple:
			language=<name_of_language>;
	}
}
\end{lstlisting}

\end{comment}


\begin{document}

%\tableofcontents

%\thispagestyle{empty}
%\newpage
% If you want to change how the subsubsection's are numbered
%\renewcommand{\thesubsection}{\thesection.\alph{subsection}.} 

%\setcounter{page}{0}
\chapter{Lessons Learned}
\section{Brandon}
I was surprised to learn how easy it is to build a very simply compiler and toy language. I though that might be the hardest part of the project - getting the scanner, parser, and codegen running. In fact, the most difficult part of the project was conceptually understanding the incredibly complex libuv library, deciding which features to incorporate in our language, and designing a plan to do that incrementally.\\\\
This fits in with my software industry experience (summer internships), in that usually the hardest part of a project is 1) conceptualizing the actual solution and 2) planning the incremental changes that add up to a significant impact. I've found if the planning is done right, implementing the incremental changes is not so hard (until it is).
\section{Somya}
When working on projects of this magnitude many skills are developed throughout the
process, moreover not only new skills are developed but also you learn how to interconnect
your different abilities and work with other's strengths to accomplish something unachievable on
your own.
Among the things learned it is worth highlighting the in depth structure of a compiler, the
scanner, the parser, the different decisions to be made when designing a language, the
differences that they bring to the language itself and to the compiler, how to express your ideas
in terms of a grammar, teamwork among others.
Systematic weekly meeting with the Professor helped us a lot and we could gauge if we were on track. Staring early is a must, it will make the difference between an excruciating workload and a
project that has incremental manageable goals that develop toward a main objective.
\section{Jeffrey}
I gained a profound respect for developers of the compilers of languages such as C, Java, Python etc... that I use regularly. I had always taken the compiler for granted and never thought about the complexity of the software that compiles/translates the code. Just to create our tiny language took the entire semester and it is extremely basic, there is no garbage collection, exceptions or even pointers, developing even a portion of a language like python and java takes not only skill and knowledge but also the ability to make good design choices in arbitrary situations. \\
I also learned the importance of teamwork, I definitely would not have been able to make pipeline alone. It took all of our individual skills and perspectives to write the language. Perspective is the key point I want to emphasize, because it was the design desisions and the critical eye of every member that, even if given an unlimited amount of time, I could not have compensated for had I designed and developed the language alone.
%\subsection{Identifiers}
%\subsubsection{}
%\subsection{subsection}
%\subsubsection{subsubsection}
\end{document}

