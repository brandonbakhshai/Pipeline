\documentclass[./Report_main.tex]{subfiles}
\begin{comment}
If you want a box around your answer and that answer is an
equation then use \boxed{$$ equation $$} 

if you want to indent a block of text:
\begin{adjustwidth}{cm of right indent}{cm of left indent}
% paragraph to be indented
\end{adjustwidth}

if you just want one indent for one line 
use \indent per intended indent per line

A sections numbers automatically, so if the number of 
the problem is out of order it would be easier to 
just indent and bold the sections and subsections
and not use the \section{} kind of commands

\newpage makes a new page

$normal math mode$
$$Special math mode$$

to include an image use
\includegraphics{image_name}
image_name is the file name (.png) without the extension. The file
name cannot have any spaces or any periods other than the one before
the file extension.

To include a codeblock use
\begin{lstlisting}
ExampleCode(blah, blah)
{
	it does tabbing and everything;
	for (coloring of major languages like java){
		add the folloing to the \lstset tuple:
			language=<name_of_language>;
	}
}
\end{lstlisting}

\end{comment}


\begin{document}

%\tableofcontents

%\thispagestyle{empty}
%\newpage
% If you want to change how the subsubsection's are numbered
%\renewcommand{\thesubsection}{\thesection.\alph{subsection}.} 

%\setcounter{page}{0}
\chapter{Test plan}
\section{Source to Target}
\textbf{Source}
\textbf{pipe.pl}
\lstinputlisting[language=Ocaml]{../sample_programs/pipe.pl}
\textbf{Target}
\textbf{pipe.c}
\lstinputlisting[language=C]{../sample_programs/pipe.c}
\textbf{Source}
\textbf{99bottles.pl}
\lstinputlisting[language=Ocaml]{../sample_programs/99blttles.pl}
\textbf{Target}
\textbf{99bottles.c}
\lstinputlisting[language=C]{../sample_programs/99blttles.c}
\section{Test Suites}
The test suites is one of the most crucial parts in the development process. We wrote a test case for every new feature we implemented and also for the corresponding semantic check. We first wrote the test case the intended to pass and then multiple cases that are suppose to fail. Since all the passing cases can be in the same program but we need to check the failing cases separately for correct error message.\\
In addition to small tests for individual feature, we also wrote some program that integrates multiple features, for example, the Fibonacci number program and the 99-bottle of beer program.\\
We managed to pass all the test cases at the end. All the test files are included in the appendix.\\
\vspace{5mm}\\
\textbf{The shell section for testing}
\section{Test Automation}
For automated testing, we wrote a python script named testall.py that will automatically compile and run all the .pl files in the test directory. The program will redirect the stdout and stderr generated by the compiler or the program to a .out file then compare the .out files with the expected output .expected file. If they are the same, the program passes the test, if not, the program fail the test. The script will mark each test case as passed or failed and print it to stdout for us to debug.\\
\section{Who Did What}
\begin{itemize}
    \item 
\end{itemize}
%\section{}
%\subsection{Identifiers}
%\subsubsection{}
%\subsection{subsection}
%\subsubsection{subsubsection}
\end{document}

